\section{Einleitung}

	Vorverstädnis, Einleitung in die Arbeit

\section{Der Text: Mk 2, 1--12}

	\subsection{Übersetzung aus dem Urtext}

		\textit{Auf der Grundlage des Textes des \textsc{Novum Testamentum Graece}\footcite{NA28} wurde der Urtext durch den Verfasser übersetzt wie folgt:}

			\textsuperscript{1} Und er kam nach einigen Tagen wieder nach Kapernaum und man hörte, dass er in einem Hause sei. \textsuperscript{2} Und es versammelten sich so viele, so daß kein Platz mehr war, auch nicht vor der Tür; und er sagte zu ihnen das Wort. \textsuperscript{3} Und sie kommen und bringen vor ihn einen Gelähmten, der von vieren getragen wurde. \textsuperscript{4} Und weil sie [ihn]\footnote{Ergänzung durch d. Vf.}  wegen der Volksmenge nicht zu ihm bringen können, deckten sie das Dach ab, wo er war, und nachdem sie es aufgegraben hatten, lassen sie das Bett hinunter, darin der Gelähmte lag. \textsuperscript{5} Und weil Jesus ihren Glauben erkannte, sagt er zu dem Gelähmten: \textit{\frqq Kind, deine Sünden werden dir erlassen.\flqq}

			\textsuperscript{6} Es saßen dort aber auch einige der Schriftgelehrten und überlegten in ihren Herzen: \textsuperscript{7} \textit{\frqq Wer ist er, dass er dies sagt? Er lästert. Wer kann Sünden erlassen außer einem, der ist Gott?\flqq} \textsuperscript{8} Und weil Jesus sogleich in seinem Geist erkannte, was sie so bei sich überlegten, sagte er zu ihnen: \textit{\frqq Was überlegt ihr in euren Herzen?} \textsuperscript{9} \textit{Was ist leichter? Dem Gelähmten zu sagen: deine Sünden werden dir erlassen? Oder zu sagen: steh auf, nimm dein Bett und geh umher?} \textsuperscript{10} \textit{Damit ihr aber wisst, dass der Menschensohn Vollmacht hat, auf der Erde die Sünden zu erlassen --}

			\textsuperscript{11} \textit{ich sage dir: steh auf, nimm dein Bett und geh in dein Haus.\flqq} \textsuperscript{12} Und er stand auf und, nachdem er sofort sein Bett genommen hatte, ging er vor allen hinaus, so daß alle staunten und den Gott verherrlichten indem sie sagten: \textit{\frqq etwas derartiges haben wir niemals gesehen.\flqq}

	\subsection{Textkritik}

		Der vorangehenden Übersetzung sowie der hier vorgelegten Exegese liegen die textkritischen Entscheidungen in Anhang 1 zugrunde.

\section{Literarkritik}

	\subsection{Kontextanalyse}

		\subsubsection{Einbettung in das Markusevangelium}

		Gliederung Mk

		Verhältnis Perikope/Mk, Mk/Evangelien, Evangelien/NT, NT/Heilige Schrift

		\subsubsection{Abgrenzung der Perikope}

			Nunmehr soll die Frage der Abgrenzung der Perikope im Markus-Evangelium behandelt werden.

			Nach hinten ist die Perikope abgegrenzt durch den erneuten Einsatz mit dem Bericht von Jesu Reise nach Kapernaum, es findet also ein Orts- und Zeitwechsel (\textit{\frqq nach einigen Tagen\flqq}) statt.

			Nach vorne ist die Perikope gleichfalls abgegrenzt durch ein Ortswechsel in V. 13: \textit{\frqq Und er ging wieder hinaus nahe des Sees...\flqq}.

	\subsection{Grammatik und Syntax}

		Wortformen, Satzarten, Grammatik

	\subsection{Textsemantik}

		Wortschatz,

	\subsection{Narrative Analyse}

		Der Handlungsablauf in der Perikope stellt sich dar wie folgt:

		\begin{enumerate}
			\item (V. 1) Jesus kommt nach Kapernaum.
			\item (V. 2) Es versammelten sich viele, Jesus predigt ihnen.
			\item (Vv. 3f.) Sie bringen ihm den Gelähmten und decken dafür das Dach ab.
			\item (V. 5) Jesus spricht und vergibt dem Gelähmten seine Sünden.
			\item (Vv. 6ff.) Die Schriftgelehrten beschuldigen Jesus, Jesus antwortet im Stil eines Streitgespräches.
			\item (Vv. 11, 12a.) Jesus spricht den Gelähmten erneut an, fordert ihn zum Aufstehen und Gehen auf; der Glähmte steht auf und geht.
			\item (V. 12b) Das Volk staunt.
		\end{enumerate}

	\subsection{Textpragmatik}

		Adressat: heidenchristliche Gemeinde

		Anlass:

		Wirkungsabsicht:

		Mittel:

		Gefühlsebene:

	\subsection{Gliederung der Perikope}

		Die Perikope wird in drei Abschnitte gegliedert:

		\paragraph{I.} Vv. 1--5: Jesu Auftritt in Kapernaum, Predigt, Vorstellung des Gelähmten und dessen Sündenvergebung.

		\paragraph{II.} Vv. 6--10: Streitgespräch mit den Pharisäern.

		\paragraph{III.} Vv. 11--12: Aufstehen des Gelähmten, Staunen des Volkes.

	\subsection{Einheitlichkeit}

		Die vorliegende Perikope ist nicht einheitlich. Schriftgelehrte!? Einfügung eines Streitgesprächs? Einfügung der Vv. 6--10 \footcite[Cf.][29f.]{Schweizer1998}

	\subsection{Quellenkritik}

		Q? Logienquelle=Jesus?

		Was ist die Zwei-Quellen-Theorie?

		In der Evangelienexegese ist ein Umstand besonders auffällig: Matthäus, Markus und Lukas stimmen in weiten Teilen inhaltlich überein -- daher man sie auch \textit{synoptische Evangelien} nennt --, wohingegen alle drei auch ihnen exklusiven Textbestand besitzen (Sondergut) bzw. gegenüber den anderen Inhalte auslassen (Lücken). Um das Verhältnis der drei synoptischen Evangelien untereinander besser beschreiben zu können, wurde die \textit{Zwei-Quellen-Theorie} entwickelt.

		GESCHICHTLICHER ABRISS

		Bei näherer Betrachtung stellt sich Mk als die größte gemeinsame Schnittmenge zwischen Mt und Lk dar und wird daher als eine Quelle betrachtet. Bei dem verbleibenden Textbestand, der nicht Inhalt von Mk ist, gibt es eine zweite Schnittmenge, die sich häufig auf Aussprüche Jesu bezieht und also \textit{Logien-Quelle} genannt wird und die zweite Quelle darstellt.

		Ganz unproblematisch ist diese Theorie jedoch nicht.

		So gibt es einerseits bei allen dreien sog. Sondergut, also, wie oben bereits angedeutet, Texte, die nur in einem Evangelium vorkommen. So z.B. Mk 2, 27 der Sabbatspruch, Mt 13, 24--30 das Unkraut unter dem Weizen oder Lk 24, 13--35 die Emmaus-Erzählung. Man müsste hier also jeweils eine weitere Sondergut-Quelle annehmen.
		Andererseits gibt es auch \textit{"minor-agreements"} zwischen Mt und Lk gegen Mk. Auch hier müsste man eine weitere Quelle annehmen.

		Es gab und gibt verschiedene Versuche, dieses Problem zu lösen, etwa durch Annahme eines Ur-Markus, der die Grundlage für den (umgearbeiteten oder mit Mk-Sondergut ergänzten) Mk sowie die heutigen Mt und Lk gewesen sein könnte.

		Jedoch ist die Zwei-Quellen-Theorie als der aktuelle Quasi-Standard in der exegetischen Wissenschaft anzusehen; wenn sie schon nicht alle Fragen beantwortet und aufgrund der vielfältigen Annahmen nicht besonders belastbar ist, so ist sie doch hilfreich, die Beziehungen der Synoptiker untereinander wenigstens ansatzweise zu verstehen.

	\subsection{Synoptischer Vergleich}

		Auf der Grundlage der Zwei-Quellen-Theorie werden die drei Synoptischen Evangelien nachfolgend einander gegenübergestellt und auch ihre Zuordnung zu ihrer jeweiligen Quelle wird vermerkt:

		\begin{table}[h]
		\begin{tabular}{|l|l|l|l|l|}
		\hline \textbf{Abschnitt} & \textbf{Mt} & \textbf{Mk} & \textbf{Lk} & \textbf{Quelle} \\\hline\hline
		          &    &    &    &    \\\hline
		          &    &    &    &    \\\hline
		          &    &    &    &    \\\hline
		          &    &    &    &    \\\hline
		          &    &    &    &    \\\hline
		          &    &    &    &    \\\hline
		          &    &    &    &    \\\hline
		          &    &    &    &    \\\hline
		          &    &    &    &    \\\hline
		          &    &    &    &    \\\hline
		          &    &    &    &    \\\hline
		          &    &    &    &    \\
							\hline
		\end{tabular}
		\end{table}

\section{Formgeschichte}

	\subsection{Traditionsgeschichte}

	\subsection{Redaktionsgeschichte}

	\subsection{Gattungsbestimmung}

		Wundergeschichte

		eingebettet: Apophthegma

		Chrie? "Wer, was, warum, gegen, ähnlich, Beispiele, Zeugen?"

\section{Begriffs- und Motivgeschichte, Historischer Zusammenhang}

	\subsection{Religionsgeschichtlicher Vergleich}

	\subsection{Historischer Zusammenhang}

		70 n Christus

		Jüdischer Krieg

		Belagerung Jerusalems

		Zerstörung des jerusalemer Tempels

		Eroberung Jerusalems

	\subsection{Begriffs- und Motivgeschichte}

		Dächer in der Antike

		Lähmung als Krankheit

		{\ibygr{lalein ton logon}} = er redet das Wort, also höchstens das Evangelium, nicht den Logos \footcite[Cf.][16, 33]{wellhausen1903Mk}

		Streitgespräch

		Wundererzählung / Heilungswunder

	\subsection{Rückfrage nach Jesus}

		Logienquelle?

\section{Interpretation}

	\subsection{Auslegung}

	\subsection{Christologie bei Markus}

	\subsection{Ausblick}

		Wert für die Verkündigung
