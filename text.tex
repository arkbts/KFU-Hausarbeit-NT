\section{Einleitung}
	
	Vorverstädnis, Einleitung in die Arbeit

\section{Der Text: Mk 2, 1--12}

	\subsection{Übersetzung aus dem Urtext}
		
		\textit{Auf der Grundlage des Textes des \textsc{Novum Testamentum Graece}\footcite{NA28} wurde der Urtext durch den Verfasser übersetzt wie folgt:}
		
			\textsuperscript{1} Und er kam nach einigen Tagen wieder nach Kapernaum und man hörte, dass er in einem Hause sei. \textsuperscript{2} Und es versammelten sich so viele, so daß kein Platz mehr war, auch nicht vor der Tür; und er sagte zu ihnen das Wort. \textsuperscript{3} Und sie kommen und bringen vor ihn einen Gelähmten, der von vieren getragen wurde. \textsuperscript{4} Und weil sie [ihn]\footnote{Ergänzung durch d. Vf.}  wegen der Volksmenge nicht zu ihm bringen können, deckten sie das Dach ab, wo er war, und nachdem sie es aufgegraben hatten, lassen sie das Bett hinunter, darin der Gelähmte lag. \textsuperscript{5} Und weil Jesus ihren Glauben erkannte, sagt er zu dem Gelähmten: \textit{\frqq Kind, deine Sünden werden dir erlassen.\flqq}
			
			\textsuperscript{6} Es saßen dort aber auch einige der Schriftgelehrten und überlegten in ihren Herzen: \textsuperscript{7} \textit{\frqq Wer ist er, dass er dies sagt? Er lästert. Wer kann Sünden erlassen außer einem, der ist Gott?\flqq} \textsuperscript{8} Und weil Jesus sogleich in seinem Geist erkannte, was sie so bei sich überlegten, sagte er zu ihnen: \textit{\frqq Was überlegt ihr in euren Herzen?} \textsuperscript{9} \textit{Was ist leichter? Dem Gelähmten zu sagen: deine Sünden werden dir erlassen? Oder zu sagen: steh auf, nimm dein Bett und geh umher?} \textsuperscript{10} \textit{Damit ihr aber wisst, dass der Menschensohn Vollmacht hat, auf der Erde die Sünden zu erlassen --}
			
			\textsuperscript{11} \textit{ich sage dir: steh auf, nimm dein Bett und geh in dein Haus.\flqq} \textsuperscript{12} Und er stand auf und, nachdem er sofort sein Bett genommen hat, ging er vor allen hinaus, so daß alle staunten und den Gott verherrlichten indem sie sagten: \textit{\frqq etwas derartiges haben wir niemals gesehen.\flqq}
		
	\subsection{Textkritik}
	
		Der vorangehenden Übersetzung sowie der hier vorgelegten Exegese liegen folgende textkritische Entscheidungen zugrunde:

		\textbf{V. 1} bietet als Varianten {\ibygr{en oikw}} (\textit{Dativ}), welche außerordentlich gut bezeugt ist, vor allem unter den ständigen Zeugen Codex Sinaiticus, der Minuskel 33 und dem Codex Vaticanus, wobei letzterem nicht nur wegen seines Alters (350 n. Chr.) die größte Bedeutung zukommt.\footcite[cf.][47]{schnelle2014} Die zweite Variante {\ibygr{eij oikon}} (\textit{Akkusativ}) wird dagegen hauptsächlich von weniger gewichtigen und auch jüngeren Textzeugen bezeugt.\footnote{5. Jahrhundert: Codex Alexandrinus, um die erste Jahrtausendwende: Minuskeln 28. 565. 579. 700. 1241. 1424. 2542. \textit{l} 2211)} Vermutlich ist diese Textvariante einem Abschreibfehler geschuldet, wobei vermutlich das {\ibygr{-n}} und das {\ibygr{-ij}} sowie das {\ibygr{-w}} und das {\ibygr{-on}} verwechselt wurden; eventuell wurde die Präpsition auch sekundär angepasst. Das Hebräische kennt außerdem keine Kasus und so könnte der Fehler auch in schlichter Unkenntnis des Unterschiedes zwischen Akkusativ und Dativ entstanden sein. Aus den vorgenannten Gründen werden die übrigen Lesarten verworfen.
		
		In \textbf{V. 2} findet sich eine durch den Codex Alexandrinus zusammen mit dem Codex Ephraemi Syri Rescriptus und diversen Minuskeln belegte Ergänzung {\ibygr{euqewj}}. Dagegen findet sich diese in den Codices Sinaiticus und Vaticanus, beides sehr alte Handschriften von hohem Textwert\footcite[cf.][46f.]{schnelle2014}, sowie die Minuskel 33 nicht. Gemäß dem Grundsatz \textit{lectio brevior potior} wird dieser Zusatz als Ausschmückung des Urtextes verworfen.
			
		\textbf{V. 3} wird von diversen Handschriften in einer anderen Satzreihenfolge dargestellt, was aber entsprechend der freien Satzstellung der griechischen Sprache grammatikalisch keinen Unterschied macht und vermutlich Abschreibefehlern zugrundeliegt. 
		
		Die Ergänzung {\ibygr{idou andrej}}, bezeugt durch die Minuskel 28, wird nicht nur aufgrund ihres geringen Alters verworfen als \textit{lectio brevior et ergo potior}.
			
		In \textbf{V. 4} findet sich statt der sehr gut im Codex Vaticanus sowie der Minuskel 33 belegten Lesart {\ibygr{prosenegkai}}, \textit{hervorbringen}, die Variante {\ibygr{proseggisai}}, \textit{sich nähern}, die im Codex Alexandrinus sowie im Codex Rescriptus und außerdem in den Minuskelfamilien \textit{f}\textsuperscript{1.13} sowie diversen weiteren Minuskeln\footnote{Nämlich: 28, 565, 579, 700, 1241, 1424, 2542.} sowie im Lektionar 2211 bezeugt ist. Diese  Variante wird wegen des relativ geringen Textwertes des Codex Alexandrinus in den Evangelien,\footcite[cf.][47]{schnelle2014} des geringen Alters der Minuskeln (datiert in das 9.-13. Jahrhundert) verworfen. Gewichtiger ist aber der Befund, dass die Variante {\ibygr{proseggisai}} vermutlich eine spätere Glättung darstellt, weil {\ibygr{prosenegkai}} kein Objekt trägt, wodurch unklar bleibt, wen die vier Männer bringen wollten; diese Variante wird gegenüber der \textit{lectio difficilior} des Textes verworfen.
		
		Die Ergänzung {\ibygr{o) Ihsouj}} wird als nur wenig bezeugte\footnote{Nämlich durch D, {\ibygr{D}}, {\ibygr{Q}} sowie die Minuskeln 700, 1424 sowie in der Vetus Latina.} und vermutlich nachträgliche Ergänzung zur Hervorhebung des Subjektes gegenüber der \textit{lectio difficilior et brevior} des Textes, welche folglich überwältigend häufig bezeugt ist, verworfen.
		
		Neben {\ibygr{o)pou}}, was sehr gut bezeugt\footnote{Nämlich im Codex Sinaiticus, Codex Vaticanus, D, L, Minuskel 892, einzelnen altlateinischen Zeugen sowie in einigen von der Vulgata abweichenden Versionen.} ist, finden sich als Varianten zum einen {\ibygr{ef w)}}, was überschaubar belegt\footnote{Nämlich in $\mathfrak{P}$\textsuperscript{84vid}, im Codex Alexandrinus sowie im Codex Rescriptus.} ist sowie {\ibygr{ef ou)}}, was wenig nur belegt ist \footnote{Und zwar in {\ibygr{Q}}, der Minuskelfamilie \textit{f}\textsuperscript{13}, den Minuskeln 33 und 565.} und schließlich {\ibygr{eij o)n}}, was lediglich im Codex Washingtonianus (4./5. Jahrhundert) belegt ist. Die Variante {\ibygr{o)pou}} sehr gut bereits seit dem 4. Jahrhundert bezeugt; alle anderen Varianten sind nur in deutlich jüngeren Texten belegt oder auch als Abschreibefehler wie z. B. {\ibygr{ou)}} gegenüber {\ibygr{o)n}} und werden daher verworfen.
		
		In \textbf{V. 5} finden sich als Varianten {\ibygr{idwn de}} und {\ibygr{kai idwn}}, was beides recht umfangreich belegt ist.\footnote{{\ibygr{idwn de}}: Codex Alexandrinus, D, K, W, {\ibygr{G}}, {\ibygr{D}}, 0130, \textit{f}\textsuperscript{1}, 579, 1242, 1424, 2542, \textit{l} 2211, Vulgata und ein teil der altlateinischen Zeugen, die gesamte syrische und ein Teil der sahidischen Überlieferung; {\ibygr{kai idwn}}: $\mathfrak{P}$\textsuperscript{88}, Codices Sinaiticus, Vaticanus, Ephraemi Syri Rescriptus, L, {\ibygr{Q}}, \textit{f}\textsuperscript{13}, 33, 565, 700, 892, den anderen beiden Synoptikern, in einigen Handschriften der sahidischen Überlieferung sowie in der bohairischen Überlieferung.} Der Vorzug wurde vorliegend aber der Variante {\ibygr{kai idwn}} gegeben, da diese einerseits auf einem älteren Textbestand beruht und andererseits auch wortgleich bei den anderen beiden Synoptikern zu finden ist\footnote{Cf. Mt 9, 4 et Lk 5, 20.}.
		
		Weiter finden sich die Varianten {\ibygr{afewntai}}, {\ibygr{afiwntai}}, {\ibygr{afiontai}}, {\ibygr{afientai}}. Nachdem die Variante {\ibygr{afientai}} jedoch gegenüber den anderen außerordentlich gut bezeugt ist\footnote{So im Codex Vaticanus, den Minuskeln 0130, 28, 33, 565, 1241 und im Lektionar 2211 sowie beinahe allen lateinischen Traditionen.} und die Variante {\ibygr{afewntai}} dem ionischen Dialekt entstammt, aber bedeutungsunerheblich ist, werden die verbliebenen beiden Varianten als Abschreibefehler, vielleicht aus Unkenntnis des Kopisten der griechischen Sprache überhaupt, verworfen.
		
		\textbf{V. 6} wird einheitlich überliefert.
		
		In \textbf{V. 7} werden die Varianten {\ibygr{o)ti}} und {\ibygr{ti}} geboten. Wohingegen die letztere überwältigend häufig bezeugt ist weichen lediglich der Codex Vaticanus sowie der Codex Koridethi davon ab und lesen {\ibygr{o)ti}}. Obschon dem Codex Vaticanus eine hohe Textqualität zugesprochen wird\footcite[cf.][XVII]{elbtk2017} ergibt sich aus seiner Verwandschaft zum Beispiel mit $\mathfrak{P}$\textsuperscript{75}\footnote{Ibidem.} aus dem 3. Jahrhundert, welches diese Lesart gerade nicht bezeugt welchem ebenfalls ein hohes Maß an Textqualität zugesprochen wird,\footcite[Cf.][46]{schnelle2014} dass die Lesart {\ibygr{ti}} wahrscheinlich die ursprünglichere ist.
		
		Außerdem steht bei den anderen beiden Synoptikern am Ende des Fragesatzes {\ibygr{blasfhmiaj}}, was aber als \textit{Lectio longior et ergo peior} bzw. als Ausschmückung verworfen wird.
		
		In \textbf{V. 8} ist das im Text stehende {\ibygr{ou(twj}} im Codex Vaticanus sowie W, {\ibygr{Q}}, in der Peschitta sowie in einzelnen sahidischen Handschriften ausgelassen. Obschon der Codex Vaticanus ein Text von hoher Qualität ist, sind die übrigen Bezeugungen später datiert. Es ist denkbar, dass diese Auslassung ein Homoioarkton ist, also wegen dem gleichen Beginn von {\ibygr{autou}}, {\ibygr{ou(twj}}, {\ibygr{e(autoij}} und {\ibygr{autoij}}. Daher wird diese Variante der Auslassung verworfen.
		
		Die Einfügung {\ibygr{autoi}}, die vielfältig bezeugt ist\footnote{Und zwar von $\mathfrak{P}$\textsuperscript{84vid}, vom Codex Alexandrinus, Codex Ephraemi Syri Rescriptus, K, {\ibygr{G}}, \textit{f}\textsuperscript{13}, von den Minuskeln 33, 1242, 1424, dem Lektionar \textit{l} 2211, dem Mehrheitstext sowie der Harklensis.}, wird als \textit{Lectio longior et ergo peior} verworfen.
		
		Die Auslassung von {\ibygr{autoij}}, welche lediglich im Codex Vaticanus, in {\ibygr{Q}} sowie ff\textsuperscript{2} bezeugt ist. Diese Variante wird wegen der schieren Überzahl der positiven Bezeugungen aus ähnlichen Gründen wie oben bei {\ibygr{ou(twj}} als Homoioarkton verworfen.
		
		In \textbf{V. 9} {\ibygr{afientai}}
		
		{\ibygr{egeire}}
		
		{\ibygr{kai aron ton krabatton sou}}
		
		{\ibygr{peripatei}}
		
		\textbf{V. 10} {\ibygr{afienai a)amartias epi thj ghj}} ...
		
		\textbf{V. 11} wird einheitlich überliefert.
		
		In \textbf{V. 12} {\ibygr{emprosqen}}
		
		{\ibygr{legontaj}}
		
		{\ibygr{ou)twj oudepote}}

\section{Synchroner Arbeitsgang, Textanalyse}

	\subsection{Abgrenzung der Perikope}
	
		Nach vorne, nach hinten
	
	\subsection{Kontextanalyse}
	
		Verhältnis Perikope/Mk, Mk/Evangelien, Evangelien/NT, NT/Heilige Schrift

	\subsection{Grammatik und Syntax}
	
		Wortformen, Satzarten, Grammatik
	
	\subsection{Textsemantik}
	
		Wortschatz, 
	
	\subsection{Narrative Analyse}
	
		Handlungsablauf

	\subsection{Textpragmatik}
	
		Adressat, Anlass, Wirkungsabsicht, Mittel, Gefühlsebene
	
	\subsection{Einheitlichkeit}
	
		Spannungen, Widersprüche etc.
		
		Schriftgelehrte!? Einfügung eines Streitgesprächs?

	\subsection{Gliederung der Perikope}
	
		(Übersetzung bereitsentsprechend gliedern)

\section{Diachroner Arbeitsgang}

	\subsection{Synoptischer Vergleich, Literarkritik}
	
		Quellenkritik
	
	\subsection{Rückfrage nach Jesus (Q)}

	\subsection{Formgeschichte}

	\subsection{Traditionsgeschichte}

	\subsection{Begriffs- und Motivgeschichte}
	
		\subsubsection{Dächer in der Antike}
	
		\subsubsection{Lähmung als Krankheit}

	\subsection{Religionsgeschichtlicher Vergleich}
	
	\subsection{Redaktionsgeschichte}

\section{Exegese}

	\subsection{Auslegung}
	
	\subsection{Ausblick}
	
		Wert für die Verkündigung