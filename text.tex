\section{Einleitung}

	Vorverstädnis, Einleitung in die Arbeit

\section{Der Text: Mk 2, 1--12}

	\subsection{Übersetzung aus dem Urtext}

		\textit{Auf der Grundlage des Textes des \textsc{Novum Testamentum Graece}\footcite{NA28} wurde der Urtext durch den Verfasser übersetzt wie folgt:}

			\textsuperscript{1} Und er kam nach einigen Tagen wieder nach Kapernaum und man hörte, dass er in einem Hause sei. \textsuperscript{2} Und es versammelten sich so viele, so daß kein Platz mehr war, auch nicht vor der Tür; und er sagte zu ihnen das Wort. \textsuperscript{3} Und sie kommen und bringen vor ihn einen Gelähmten, der von vieren getragen wurde. \textsuperscript{4} Und weil sie [ihn]\footnote{Ergänzung durch d. Vf.}  wegen der Volksmenge nicht zu ihm bringen können, deckten sie das Dach ab, wo er war, und nachdem sie es aufgegraben hatten, lassen sie das Bett hinunter, darin der Gelähmte lag. \textsuperscript{5} Und weil Jesus ihren Glauben erkannte, sagt er zu dem Gelähmten: \textit{\frqq Kind, deine Sünden werden dir erlassen.\flqq}

			\textsuperscript{6} Es saßen dort aber auch einige der Schriftgelehrten und überlegten in ihren Herzen: \textsuperscript{7} \textit{\frqq Wer ist er, dass er dies sagt? Er lästert. Wer kann Sünden erlassen außer einem, der ist Gott?\flqq} \textsuperscript{8} Und weil Jesus sogleich in seinem Geist erkannte, was sie so bei sich überlegten, sagte er zu ihnen: \textit{\frqq Was überlegt ihr in euren Herzen?} \textsuperscript{9} \textit{Was ist leichter? Dem Gelähmten zu sagen: deine Sünden werden dir erlassen? Oder zu sagen: steh auf, nimm dein Bett und geh umher?} \textsuperscript{10} \textit{Damit ihr aber wisst, dass der Menschensohn Vollmacht hat, auf der Erde die Sünden zu erlassen --}

			\textsuperscript{11} \textit{ich sage dir: steh auf, nimm dein Bett und geh in dein Haus.\flqq} \textsuperscript{12} Und er stand auf und, nachdem er sofort sein Bett genommen hatte, ging er vor allen hinaus, so daß alle staunten und den Gott verherrlichten indem sie sagten: \textit{\frqq etwas derartiges haben wir niemals gesehen.\flqq}

	\subsection{Textkritik}

		Der vorangehenden Übersetzung sowie der hier vorgelegten Exegese liegen folgende textkritische Entscheidungen zugrunde:

		\textbf{V. 1} bietet als Varianten {\ibygr{en oikw}} (\textit{Dativ}), welche außerordentlich gut bezeugt ist, vor allem unter den ständigen Zeugen Codex Sinaiticus, der Minuskel 33 und dem Codex Vaticanus, wobei letzterem nicht nur wegen seines Alters (350 n. Chr.) die größte Bedeutung zukommt.\footcite[cf.][47]{schnelle2014} Die zweite Variante {\ibygr{eij oikon}} (\textit{Akkusativ}) wird dagegen hauptsächlich von weniger gewichtigen und auch jüngeren Textzeugen bezeugt.\footnote{5. Jahrhundert: Codex Alexandrinus, um die erste Jahrtausendwende: Minuskeln 28. 565. 579. 700. 1241. 1424. 2542. \textit{l} 2211)} Vermutlich ist diese Textvariante einem Abschreibfehler geschuldet, wobei vermutlich das {\ibygr{-n}} und das {\ibygr{-ij}} sowie das {\ibygr{-w}} und das {\ibygr{-on}} verwechselt wurden; eventuell wurde die Präpsition auch sekundär angepasst. Das Hebräische kennt außerdem keine Kasus und so könnte der Fehler auch in schlichter Unkenntnis des Unterschiedes zwischen Akkusativ und Dativ entstanden sein. Aus den vorgenannten Gründen werden die übrigen Lesarten verworfen.

		In \textbf{V. 2} findet sich eine durch den Codex Alexandrinus zusammen mit dem Codex Ephraemi Syri Rescriptus und diversen Minuskeln belegte Ergänzung {\ibygr{euqewj}}. Dagegen findet sich diese in den Codices Sinaiticus und Vaticanus, beides sehr alte Handschriften von hohem Textwert\footcite[cf.][46f.]{schnelle2014}, sowie die Minuskel 33 nicht. Gemäß dem Grundsatz \textit{lectio brevior potior} wird dieser Zusatz als Ausschmückung des Urtextes verworfen.

		\textbf{V. 3} wird von diversen Handschriften in einer anderen Satzreihenfolge dargestellt, was aber entsprechend der freien Satzstellung der griechischen Sprache grammatikalisch keinen Unterschied macht und vermutlich Abschreibefehlern zugrundeliegt.

		Die Ergänzung {\ibygr{idou andrej}}, bezeugt durch die Minuskel 28, wird nicht nur aufgrund ihres geringen Alters verworfen als \textit{lectio brevior et ergo potior}.

		In \textbf{V. 4} findet sich statt der sehr gut im Codex Vaticanus sowie der Minuskel 33 belegten Lesart {\ibygr{prosenegkai}}, \textit{hervorbringen}, die Variante {\ibygr{proseggisai}}, \textit{sich nähern}, die im Codex Alexandrinus sowie im Codex Rescriptus und außerdem in den Minuskelfamilien \textit{f}\textsuperscript{1.13} sowie diversen weiteren Minuskeln\footnote{Nämlich: 28, 565, 579, 700, 1241, 1424, 2542.} sowie im Lektionar 2211 bezeugt ist. Diese  Variante wird wegen des relativ geringen Textwertes des Codex Alexandrinus in den Evangelien,\footcite[cf.][47]{schnelle2014} des geringen Alters der Minuskeln (datiert in das 9.-13. Jahrhundert) verworfen. Gewichtiger ist aber der Befund, dass die Variante {\ibygr{proseggisai}} vermutlich eine spätere Glättung darstellt, weil {\ibygr{prosenegkai}} kein Objekt trägt, wodurch unklar bleibt, wen die vier Männer bringen wollten; diese Variante wird gegenüber der \textit{lectio difficilior} des Textes verworfen.

		Die Ergänzung {\ibygr{o) Ihsouj}} wird als nur wenig bezeugte\footnote{Nämlich durch D, {\ibygr{D}}, {\ibygr{Q}} sowie die Minuskeln 700, 1424 sowie in der Vetus Latina.} und vermutlich nachträgliche Ergänzung zur Hervorhebung des Subjektes gegenüber der \textit{lectio difficilior et brevior} des Textes, welche folglich überwältigend häufig bezeugt ist, verworfen.

		Neben {\ibygr{o)pou}}, was sehr gut bezeugt\footnote{Nämlich im Codex Sinaiticus, Codex Vaticanus, D, L, Minuskel 892, einzelnen altlateinischen Zeugen sowie in einigen von der Vulgata abweichenden Versionen.} ist, finden sich als Varianten zum einen {\ibygr{ef w)}}, was überschaubar belegt\footnote{Nämlich in $\mathfrak{P}$\textsuperscript{84vid}, im Codex Alexandrinus sowie im Codex Rescriptus.} ist sowie {\ibygr{ef ou)}}, was wenig nur belegt ist \footnote{Und zwar in {\ibygr{Q}}, der Minuskelfamilie \textit{f}\textsuperscript{13}, den Minuskeln 33 und 565.} und schließlich {\ibygr{eij o)n}}, was lediglich im Codex Washingtonianus (4./5. Jahrhundert) belegt ist. Die Variante {\ibygr{o)pou}} sehr gut bereits seit dem 4. Jahrhundert bezeugt; alle anderen Varianten sind nur in deutlich jüngeren Texten belegt oder auch als Abschreibefehler wie z. B. {\ibygr{ou)}} gegenüber {\ibygr{o)n}} und werden daher verworfen.

		In \textbf{V. 5} finden sich als Varianten {\ibygr{idwn de}} und {\ibygr{kai idwn}}, was beides recht umfangreich belegt ist.\footnote{{\ibygr{idwn de}}: Codex Alexandrinus, D, K, W, {\ibygr{G}}, {\ibygr{D}}, 0130, \textit{f}\textsuperscript{1}, 579, 1242, 1424, 2542, \textit{l} 2211, Vulgata und ein teil der altlateinischen Zeugen, die gesamte syrische und ein Teil der sahidischen Überlieferung; {\ibygr{kai idwn}}: $\mathfrak{P}$\textsuperscript{88}, Codices Sinaiticus, Vaticanus, Ephraemi Syri Rescriptus, L, {\ibygr{Q}}, \textit{f}\textsuperscript{13}, 33, 565, 700, 892, den anderen beiden Synoptikern, in einigen Handschriften der sahidischen Überlieferung sowie in der bohairischen Überlieferung.} Der Vorzug wurde vorliegend aber der Variante {\ibygr{kai idwn}} gegeben, da diese einerseits auf einem älteren Textbestand beruht und andererseits auch wortgleich bei den anderen beiden Synoptikern zu finden ist\footnote{Cf. Mt 9, 4 et Lk 5, 20.}.

		Weiter finden sich die Varianten {\ibygr{afewntai}}, {\ibygr{afiwntai}}, {\ibygr{afiontai}}, {\ibygr{afientai}}. Nachdem die Variante {\ibygr{afientai}} jedoch gegenüber den anderen außerordentlich gut bezeugt ist\footnote{So im Codex Vaticanus, den Minuskeln 0130, 28, 33, 565, 1241 und im Lektionar 2211 sowie beinahe allen lateinischen Traditionen.} und die Variante {\ibygr{afewntai}} dem ionischen Dialekt entstammt, aber bedeutungsunerheblich ist, werden die verbliebenen beiden Varianten als Abschreibefehler, vielleicht aus Unkenntnis des Kopisten der griechischen Sprache überhaupt, verworfen.

		In \textbf{V. 7} werden die Varianten {\ibygr{o)ti}} und {\ibygr{ti}} geboten. Wohingegen die letztere überwältigend häufig bezeugt ist weichen lediglich der Codex Vaticanus sowie der Codex Coridethianus davon ab und lesen {\ibygr{o)ti}}. Obschon dem Codex Vaticanus eine hohe Textqualität zugesprochen wird\footcite[cf.][XVII]{elbtk2017} ergibt sich aus seiner Verwandschaft zum Beispiel mit $\mathfrak{P}$\textsuperscript{75}\footnote{Ibidem.} aus dem 3. Jahrhundert, welches diese Lesart gerade nicht bezeugt und welchem ebenfalls ein hohes Maß an Textqualität zugesprochen wird,\footcite[Cf.][46]{schnelle2014} dass die Lesart {\ibygr{ti}} wahrscheinlich die ursprünglichere ist.

		Außerdem steht bei den anderen beiden Synoptikern am Ende des Fragesatzes {\ibygr{blasfhmiaj}}, was aber als \textit{Lectio longior et ergo peior} bzw. als Ausschmückung verworfen wird.

		In \textbf{V. 8} ist das im Text stehende {\ibygr{ou(twj}} im Codex Vaticanus sowie W, {\ibygr{Q}}, in der Peschitta sowie in einzelnen sahidischen Handschriften ausgelassen. Obschon der Codex Vaticanus ein Text von hoher Qualität ist, sind die übrigen Bezeugungen später datiert. Es ist denkbar, dass diese Auslassung ein Homoioarkton ist, also wegen dem gleichen Beginn von {\ibygr{autou}}, {\ibygr{ou(twj}}, {\ibygr{e(autoij}} und {\ibygr{autoij}}. Daher wird diese Variante der Auslassung verworfen.

		Die Einfügung {\ibygr{autoi}}, die vielfältig bezeugt ist\footnote{Und zwar von $\mathfrak{P}$\textsuperscript{84vid}, vom Codex Alexandrinus, Codex Ephraemi Syri Rescriptus, K, {\ibygr{G}}, \textit{f}\textsuperscript{13}, von den Minuskeln 33, 1242, 1424, dem Lektionar \textit{l} 2211, dem Mehrheitstext sowie der Harklensis.}, wird als \textit{Lectio longior et ergo peior} verworfen.

		Die Auslassung von {\ibygr{autoij}} ist lediglich im Codex Vaticanus, in {\ibygr{Q}} sowie ff\textsuperscript{2} bezeugt. Diese Variante wird wegen der schieren Überzahl der positiven Bezeugungen aus ähnlichen Gründen wie oben bei {\ibygr{ou(twj}} als Homoioarkton verworfen.

		In \textbf{V. 9} steht anstelle von {\ibygr{afientai}} in einigen Handschriften {\ibygr{afewntai}}. Diese Variante wird aus denselben Gründen der textkritischen Entscheidung oben in Vers 5 als Abschreibe- bzw. als Hörfehler verworfen.

		Statt {\ibygr{egeire}} findet sich überschaubar bezeugt {\ibygr{egeirou}}. Dies könnte ebenfalls als Abschreibe- bzw. Hörfehler sein. Jedoch ist {\ibygr{egeirou}} auch zugleich der attische Imperativ, was aber, wie oben bei Vers 6 dargelegt, hier ebenfalls bedeutungsunerheblich ist und zugleich auch der Sprachunkenntnis des Kopisten geschuldet sein kann. Die Variante wird daher verworfen.

		{\ibygr{kai aron ton krabatton sou}} wird bisweilen in einer anderen Reihenfolge bezeugt, was aber ohne Bedeutung ist. Teilweise fällt das {\ibygr{kai}} vermutlich in Angleichung an Mt 9,6 fort\footcite[Cf.][51]{schnelle2014}\footnote{Unter anderem \textit{33}, aber auch im Codex Rescritpus oder auch im Codex Bezae Cantabrigiensis}. In der Minuskel 700 fällt gar {\ibygr{aron}} fort, dies aber vermutlich als Homoioteleuton.

		Statt {\ibygr{peripatei}}, \textit{geh umher}, findet sich {\ibygr{u)page}}, \textit{geh weg}, \footnote{So in $\mathfrak{P}$\textsuperscript{88}, Codex Sinaiticus und Weitere.} was sich ebenfalls in Mt 9,6 findet, weshalb vorliegend erneut eine sekundäre Angleichung an Mt angenommen, diese Lesart mithin verworfen wird.\footcite[Cf.][51]{schnelle2014}

		\textbf{V. 10} {\ibygr{afienai a)martias epi thj ghj}} wird teilweise in anderer Reihenfolge bezeugt. Durch die Umstellung von {\ibygr{afienai a)martias} an den Schluss des Satzteiles wird {\ibygr{epi thj ghj}}, also die universelle Vollmacht Jesu, betont: \textit{..., dass der Menschensohn \textbf{auf der Erde} Vollmacht hat, die Sünden zu erlassen.}\footnote{So bezeugen  $\mathfrak{P}$\textsuperscript{88}, Codex Sinaiticus, Codex Rescriptus, Codex Bezae Cantabrigiensis, einige Minuskelhandschriften, darunter \textit{33}, die lateinische Vulgata, die syrische Peschitta und weitere Übersetzungen.} Die Stellung {\ibygr{afienai epi thj ghj a)martias}} betont gleichfalls Jesu universelle Vollmacht (dies wird bei der Interpretation noch eine Rolle spielen).\footnote{So bezeugen Codex Alexandrinus sowie \textit{f}\textsuperscript{1,13}.} In wenigen Handschriften \footnote{Nämlich W, b und q.} wird {\ibygr{epi thj ghj}} gar ganz ausgelassen, was aber vermutlich ein Homoioteleuton mit {\ibygr{legei}} darstellt. Vorliegend wird aber der im Text nachgewiesenen Reihung der Vorzug gegeben, da der diese Textfassung belegende Codex Vaticanus der wesentlich ältere Text gegenüber den anderen Varianten ist.\footcite[Cf.][47]{schnelle2014} Daher werden die Varianten verworfen.

		In \textbf{V. 12} wird anstelle von {\ibygr{emprosqen}} sinnähnlich bezeugt {\ibygr{enantion}}\footnote{So die Codices Alexandrinus, Rescriptus et Bezae Cantabrigiensis, die Minuskelfamilien \textit{f}\textsuperscript{1.13} und das Lektionar 2211.} sowie {\ibygr{enwpion}}\footnote{So der Codex Coridethianus aber auch u.a. die Minuskeln 28, 33.}.
		{\ibygr{enwpion}} könnte ein Hör- bzw. Abschreibefehler von {\ibygr{enantion}} sein. Dies wird auch dadurch gestützt, dass die Überlieferungen von {\ibygr{enwpion}} allesamt um das 9. Jahrhundert datiert werden, diejenigen von {\ibygr{enantion}} sind jedoch deutlich älter, nämlich durch die Codices Alexandrinus Rescriptus et Bezae Cantabrigiensis auf das 4. Jahrhundert datiert.
		{\ibygr{enantion}} ist gegenüber {\ibygr{emprosqen}}\footnote{So u.a. belegt durch die Codixes Sinaiticus et Vaticanus sowie die Minuskeln 579, 700 und 892.} deutlich schwächer belegt, da der Codex Vaticanus die älteste Pergamenthandschrift ist, die anderen Textzeugen sind neueren Datums. Von der inneren Textkritik her ist die Variante {\ibygr{emprosqen}} wahrscheinlicher, bedeutet es doch \textit{vor} -- wohingegen {\ibygr{enantion}} \textit{gegenüber} (mit einer Nuance der Feindlichkeit oder Gegnerschaft) meint.
		Daher werden beide Varianten verworfen.

		{\ibygr{legontaj}} wird in wenigen Handschriften ausgelassen, in der überwiegenden Mehrheit und quer durch alle Handschriftgattungen jedoch ist es belegt. Daher wird diese Variante verworfen.

		Die teilweise überlieferte Umstellung von {\ibygr{ou)twj oudepote}} nach {\ibygr{oudepote ou)twj}} wird als Abschreibefehler verworfen.

\section{Literarkritik}

	\subsection{Kontextanalyse}

		\subsubsection{Einbettung in das Markusevangelium}

		Gliederung Mk

		Verhältnis Perikope/Mk, Mk/Evangelien, Evangelien/NT, NT/Heilige Schrift

		\subsubsection{Abgrenzung der Perikope}

			Nunmehr soll die Frage der Abgrenzung der Perikope im Markus-Evangelium behandelt werden.

			Nach hinten ist die Perikope abgegrenzt durch den erneuten Einsatz mit dem Bericht von Jesu Reise nach Kapernaum, es findet also ein Orts- und Zeitwechsel (\textit{\frqq nach einigen Tagen\flqq}) statt.

			Nach vorne ist die Perikope gleichfalls abgegrenzt durch ein Ortswechsel in V. 13: \textit{\frqq Und er ging wieder hinaus nahe des Sees...\flqq}.

	\subsection{Grammatik und Syntax}

		Wortformen, Satzarten, Grammatik

	\subsection{Textsemantik}

		Wortschatz,

	\subsection{Narrative Analyse}

		Der Handlungsablauf in der Perikope stellt sich dar wie folgt:

		\begin{enumerate}
			\item (V. 1) Jesus kommt nach Kapernaum.
			\item (V. 2) Es versammelten sich viele, Jesus predigt ihnen.
			\item (Vv. 3f.) Sie bringen ihm den Gelähmten und decken dafür das Dach ab.
			\item (V. 5) Jesus spricht und vergibt dem Gelähmten seine Sünden.
			\item (Vv. 6ff.) Die Schriftgelehrten beschuldigen Jesus, Jesus antwortet im Stil eines Streitgespräches.
			\item (Vv. 11, 12a.) Jesus spricht den Gelähmten erneut an, fordert ihn zum Aufstehen und Gehen auf; der Glähmte steht auf und geht.
			\item (V. 12b) Das Volk staunt.
		\end{enumerate}}

	\subsection{Textpragmatik}

		Adressat: heidenchristliche Gemeinde

		Anlass:

		Wirkungsabsicht:

		Mittel:

		Gefühlsebene:

	\subsection{Gliederung der Perikope}

		Die Perikope wird in drei Abschnitte gegliedert:

		\paragraph{I.} Vv. 1--5: Jesu Auftritt in Kapernaum, Predigt, Vorstellung des Gelähmten und dessen Sündenvergebung.

		\paragraph{II.} Vv. 6--10: Streitgespräch mit den Pharisäern.

		\paragraph{III.} Vv. 11--12: Aufstehen des Gelähmten, Staunen des Volkes.

	\subsection{Einheitlichkeit}

		Die vorliegende Perikope ist nicht einheitlich. Schriftgelehrte!? Einfügung eines Streitgesprächs? Einfügung der Vv. 6--10 \footcite[Cf.][29f.]{Schweizer1998}

	\subsection{Quellenkritik}

		Q? Logienquelle=Jesus?

		Was ist die Zwei-Quellen-Theorie?

		In der Evangelienexegese ist ein Umstand besonders auffällig: Matthäus, Markus und Lukas stimmen in weiten Teilen inhaltlich überein -- daher man sie auch \textit{synoptische Evangelien} nennt --, wohingegen alle drei auch ihnen exklusiven Textbestand besitzen (Sondergut) bzw. gegenüber den anderen Inhalte auslassen (Lücken). Um das Verhältnis der drei synoptischen Evangelien untereinander besser beschreiben zu können, wurde die \textit{Zwei-Quellen-Theorie} entwickelt.

		GESCHICHTLICHER ABRISS

		Bei näherer Betrachtung stellt sich Mk als die größte gemeinsame Schnittmenge zwischen Mt und Lk dar und wird daher als eine Quelle betrachtet. Bei dem verbleibenden Textbestand, der nicht Inhalt von Mk ist, gibt es eine zweite Schnittmenge, die sich häufig auf Aussprüche Jesu bezieht und also \textit{Logien-Quelle} genannt wird und die zweite Quelle darstellt.

		Ganz unproblematisch ist diese Theorie jedoch nicht.

		So gibt es einerseits bei allen dreien sog. Sondergut, also, wie oben bereits angedeutet, Texte, die nur in einem Evangelium vorkommen. So z.B. Mk 2, 27 der Sabbatspruch, Mt 13, 24--30 das Unkraut unter dem Weizen oder Lk 24, 13--35 die Emmaus-Erzählung. Man müsste hier also jeweils eine weitere Sondergut-Quelle annehmen.
		Andererseits gibt es auch \textit{"minor-agreements"} zwischen Mt und Lk gegen Mk. Auch hier müsste man eine weitere Quelle annehmen.

		Es gab und gibt verschiedene Versuche, dieses Problem zu lösen, etwa durch Annahme eines Ur-Markus, der die Grundlage für den (umgearbeiteten oder mit Mk-Sondergut ergänzten) Mk sowie die heutigen Mt und Lk gewesen sein könnte.

		Jedoch ist die Zwei-Quellen-Theorie als der aktuelle Quasi-Standard in der exegetischen Wissenschaft anzusehen; wenn sie schon nicht alle Fragen beantwortet und aufgrund der vielfältigen Annahmen nicht besonders belastbar ist, so ist sie doch hilfreich, die Beziehungen der Synoptiker untereinander wenigstens ansatzweise zu verstehen.

	\subsection{Synoptischer Vergleich}

		Auf der Grundlage der Zwei-Quellen-Theorie werden die drei Synoptischen Evangelien nachfolgend einander gegenübergestellt und auch ihre Zuordnung zu ihrer jeweiligen Quelle wird vermerkt:

		\begin{table}[h]
		\begin{tabular}{|l|l|l|l|l|}
		\hline \textbf{Abschnitt} & \textbf{Mt} & \textbf{Mk} & \textbf{Lk} & \textbf{Quelle} \\\hline\hline
		          &    &    &    &    \\\hline
		          &    &    &    &    \\\hline
		          &    &    &    &    \\\hline
		          &    &    &    &    \\\hline
		          &    &    &    &    \\\hline
		          &    &    &    &    \\\hline
		          &    &    &    &    \\\hline
		          &    &    &    &    \\\hline
		          &    &    &    &    \\\hline
		          &    &    &    &    \\\hline
		          &    &    &    &    \\\hline
		          &    &    &    &    \\
							\hline
		\end{tabular}
		\end{table}

\section{Formgeschichte}

	\subsection{Traditionsgeschichte}

	\subsection{Redaktionsgeschichte}

	\subsection{Gattungsbestimmung}

		Wundergeschichte

		eingebettet: Apophthegma

		Chrie? "Wer, was, warum, gegen, ähnlich, Beispiele, Zeugen?"

\section{Begriffs- und Motivgeschichte, Historischer Zusammenhang}

	\subsection{Religionsgeschichtlicher Vergleich}

	\subsection{Historischer Zusammenhang}

		70 n Christus

		Jüdischer Krieg

		Belagerung Jerusalems

		Zerstörung des jerusalemer Tempels

		Eroberung Jerusalems

	\subsection{Begriffs- und Motivgeschichte}

		Dächer in der Antike

		Lähmung als Krankheit

		{\ibygr{lalein ton logon}} = er redet das Wort, also höchstens das Evangelium, nicht den Logos \footcite[Cf.][16, 33]{wellhausen1903Mk}

		Streitgespräch

		Wundererzählung / Heilungswunder

	\subsection{Rückfrage nach Jesus}

		Logienquelle?

\section{Interpretation}

	\subsection{Auslegung}

	\subsection{Christologie bei Markus}

	\subsection{Ausblick}

		Wert für die Verkündigung
